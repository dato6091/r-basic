\documentclass[]{article}
\usepackage{lmodern}
\usepackage{amssymb,amsmath}
\usepackage{ifxetex,ifluatex}
\usepackage{fixltx2e} % provides \textsubscript
\ifnum 0\ifxetex 1\fi\ifluatex 1\fi=0 % if pdftex
  \usepackage[T1]{fontenc}
  \usepackage[utf8]{inputenc}
\else % if luatex or xelatex
  \ifxetex
    \usepackage{mathspec}
  \else
    \usepackage{fontspec}
  \fi
  \defaultfontfeatures{Ligatures=TeX,Scale=MatchLowercase}
\fi
% use upquote if available, for straight quotes in verbatim environments
\IfFileExists{upquote.sty}{\usepackage{upquote}}{}
% use microtype if available
\IfFileExists{microtype.sty}{%
\usepackage{microtype}
\UseMicrotypeSet[protrusion]{basicmath} % disable protrusion for tt fonts
}{}
\usepackage[margin=1in]{geometry}
\usepackage{hyperref}
\hypersetup{unicode=true,
            pdftitle={Tarea1},
            pdfauthor={Daniel\_Torrejon},
            pdfborder={0 0 0},
            breaklinks=true}
\urlstyle{same}  % don't use monospace font for urls
\usepackage{color}
\usepackage{fancyvrb}
\newcommand{\VerbBar}{|}
\newcommand{\VERB}{\Verb[commandchars=\\\{\}]}
\DefineVerbatimEnvironment{Highlighting}{Verbatim}{commandchars=\\\{\}}
% Add ',fontsize=\small' for more characters per line
\usepackage{framed}
\definecolor{shadecolor}{RGB}{248,248,248}
\newenvironment{Shaded}{\begin{snugshade}}{\end{snugshade}}
\newcommand{\AlertTok}[1]{\textcolor[rgb]{0.94,0.16,0.16}{#1}}
\newcommand{\AnnotationTok}[1]{\textcolor[rgb]{0.56,0.35,0.01}{\textbf{\textit{#1}}}}
\newcommand{\AttributeTok}[1]{\textcolor[rgb]{0.77,0.63,0.00}{#1}}
\newcommand{\BaseNTok}[1]{\textcolor[rgb]{0.00,0.00,0.81}{#1}}
\newcommand{\BuiltInTok}[1]{#1}
\newcommand{\CharTok}[1]{\textcolor[rgb]{0.31,0.60,0.02}{#1}}
\newcommand{\CommentTok}[1]{\textcolor[rgb]{0.56,0.35,0.01}{\textit{#1}}}
\newcommand{\CommentVarTok}[1]{\textcolor[rgb]{0.56,0.35,0.01}{\textbf{\textit{#1}}}}
\newcommand{\ConstantTok}[1]{\textcolor[rgb]{0.00,0.00,0.00}{#1}}
\newcommand{\ControlFlowTok}[1]{\textcolor[rgb]{0.13,0.29,0.53}{\textbf{#1}}}
\newcommand{\DataTypeTok}[1]{\textcolor[rgb]{0.13,0.29,0.53}{#1}}
\newcommand{\DecValTok}[1]{\textcolor[rgb]{0.00,0.00,0.81}{#1}}
\newcommand{\DocumentationTok}[1]{\textcolor[rgb]{0.56,0.35,0.01}{\textbf{\textit{#1}}}}
\newcommand{\ErrorTok}[1]{\textcolor[rgb]{0.64,0.00,0.00}{\textbf{#1}}}
\newcommand{\ExtensionTok}[1]{#1}
\newcommand{\FloatTok}[1]{\textcolor[rgb]{0.00,0.00,0.81}{#1}}
\newcommand{\FunctionTok}[1]{\textcolor[rgb]{0.00,0.00,0.00}{#1}}
\newcommand{\ImportTok}[1]{#1}
\newcommand{\InformationTok}[1]{\textcolor[rgb]{0.56,0.35,0.01}{\textbf{\textit{#1}}}}
\newcommand{\KeywordTok}[1]{\textcolor[rgb]{0.13,0.29,0.53}{\textbf{#1}}}
\newcommand{\NormalTok}[1]{#1}
\newcommand{\OperatorTok}[1]{\textcolor[rgb]{0.81,0.36,0.00}{\textbf{#1}}}
\newcommand{\OtherTok}[1]{\textcolor[rgb]{0.56,0.35,0.01}{#1}}
\newcommand{\PreprocessorTok}[1]{\textcolor[rgb]{0.56,0.35,0.01}{\textit{#1}}}
\newcommand{\RegionMarkerTok}[1]{#1}
\newcommand{\SpecialCharTok}[1]{\textcolor[rgb]{0.00,0.00,0.00}{#1}}
\newcommand{\SpecialStringTok}[1]{\textcolor[rgb]{0.31,0.60,0.02}{#1}}
\newcommand{\StringTok}[1]{\textcolor[rgb]{0.31,0.60,0.02}{#1}}
\newcommand{\VariableTok}[1]{\textcolor[rgb]{0.00,0.00,0.00}{#1}}
\newcommand{\VerbatimStringTok}[1]{\textcolor[rgb]{0.31,0.60,0.02}{#1}}
\newcommand{\WarningTok}[1]{\textcolor[rgb]{0.56,0.35,0.01}{\textbf{\textit{#1}}}}
\usepackage{graphicx,grffile}
\makeatletter
\def\maxwidth{\ifdim\Gin@nat@width>\linewidth\linewidth\else\Gin@nat@width\fi}
\def\maxheight{\ifdim\Gin@nat@height>\textheight\textheight\else\Gin@nat@height\fi}
\makeatother
% Scale images if necessary, so that they will not overflow the page
% margins by default, and it is still possible to overwrite the defaults
% using explicit options in \includegraphics[width, height, ...]{}
\setkeys{Gin}{width=\maxwidth,height=\maxheight,keepaspectratio}
\IfFileExists{parskip.sty}{%
\usepackage{parskip}
}{% else
\setlength{\parindent}{0pt}
\setlength{\parskip}{6pt plus 2pt minus 1pt}
}
\setlength{\emergencystretch}{3em}  % prevent overfull lines
\providecommand{\tightlist}{%
  \setlength{\itemsep}{0pt}\setlength{\parskip}{0pt}}
\setcounter{secnumdepth}{0}
% Redefines (sub)paragraphs to behave more like sections
\ifx\paragraph\undefined\else
\let\oldparagraph\paragraph
\renewcommand{\paragraph}[1]{\oldparagraph{#1}\mbox{}}
\fi
\ifx\subparagraph\undefined\else
\let\oldsubparagraph\subparagraph
\renewcommand{\subparagraph}[1]{\oldsubparagraph{#1}\mbox{}}
\fi

%%% Use protect on footnotes to avoid problems with footnotes in titles
\let\rmarkdownfootnote\footnote%
\def\footnote{\protect\rmarkdownfootnote}

%%% Change title format to be more compact
\usepackage{titling}

% Create subtitle command for use in maketitle
\providecommand{\subtitle}[1]{
  \posttitle{
    \begin{center}\large#1\end{center}
    }
}

\setlength{\droptitle}{-2em}

  \title{Tarea1}
    \pretitle{\vspace{\droptitle}\centering\huge}
  \posttitle{\par}
    \author{Daniel\_Torrejon}
    \preauthor{\centering\large\emph}
  \postauthor{\par}
      \predate{\centering\large\emph}
  \postdate{\par}
    \date{15/11/2019}


\begin{document}
\maketitle

\hypertarget{tarea-1}{%
\section{TAREA 1}\label{tarea-1}}

\hypertarget{pregunta-1}{%
\subsection{PREGUNTA 1:}\label{pregunta-1}}

Si hubiéramos empezado a contar segundos a partir de las 12 campanadas
que marcan el inicio de 2018, ¿a qué hora de qué día de qué año
llegaríamos a los 250 millones de segundos? ¡Cuidado con los años
bisiestos!

\begin{Shaded}
\begin{Highlighting}[]
\NormalTok{seg =}\StringTok{ }\DecValTok{250} \OperatorTok{*}\StringTok{ }\DecValTok{10}\OperatorTok{^}\DecValTok{6}
\NormalTok{dias =}\StringTok{ }\NormalTok{seg}\OperatorTok{/}\NormalTok{(}\DecValTok{3600}\OperatorTok{*}\DecValTok{24}\NormalTok{)}
\NormalTok{anyos =}\StringTok{ }\NormalTok{dias}\OperatorTok{/}\FloatTok{365.25}
\NormalTok{anyos}
\end{Highlighting}
\end{Shaded}

\begin{verbatim}
  [1] 7.922022
\end{verbatim}

Sabemos que han pasado 7 años, dos de los cuales serían bisiestos (2020
y 2024), por lo tanto el año actual sería:

\begin{Shaded}
\begin{Highlighting}[]
\NormalTok{anyo =}\StringTok{ }\DecValTok{2018} \OperatorTok{+}\StringTok{ }\DecValTok{7} \CommentTok{# = 2025}
\end{Highlighting}
\end{Shaded}

7 años, en días serían:

\begin{Shaded}
\begin{Highlighting}[]
\NormalTok{dias_}\DecValTok{7}\NormalTok{_a =}\StringTok{ }\DecValTok{7} \OperatorTok{*}\StringTok{ }\DecValTok{365} \OperatorTok{+}\StringTok{ }\DecValTok{2} \CommentTok{# Añadimos dos días por los años bisiestos}
\NormalTok{dias_}\DecValTok{7}\NormalTok{_a}
\end{Highlighting}
\end{Shaded}

\begin{verbatim}
## [1] 2557
\end{verbatim}

Restamos los días transcurridos en 7 años al total de días
transcurridos:

\begin{Shaded}
\begin{Highlighting}[]
\NormalTok{dias_a2025 =}\StringTok{ }\NormalTok{dias }\OperatorTok{-}\StringTok{ }\NormalTok{dias_}\DecValTok{7}\NormalTok{_a}
\NormalTok{dias_a2025}
\end{Highlighting}
\end{Shaded}

\begin{verbatim}
  [1] 336.5185
\end{verbatim}

Con lo que obtenemos el total de días transcurridos del año 2025:
336,5185. Para obtener la fecha, restamos el total de días transcurridos
a los días del año (365 al no ser bisiesto):

\begin{Shaded}
\begin{Highlighting}[]
\NormalTok{d =}\StringTok{ }\DecValTok{365} \OperatorTok{-}\StringTok{ }\NormalTok{dias_a2025}
\NormalTok{d}
\end{Highlighting}
\end{Shaded}

\begin{verbatim}
  [1] 28.48148
\end{verbatim}

Obtenemos que faltan 28,48 días para que finalice el año. Por lo tanto,
como Diciembre tiene 31 días, estamos en Diciembre, concretamente en el
día:

\begin{Shaded}
\begin{Highlighting}[]
\NormalTok{dia_act =}\StringTok{ }\DecValTok{31}\OperatorTok{-}\NormalTok{d}
\NormalTok{dia_act =}\StringTok{ }\KeywordTok{floor}\NormalTok{(dia_act)}
\NormalTok{dia_act}
\end{Highlighting}
\end{Shaded}

\begin{verbatim}
  [1] 2
\end{verbatim}

La fecha es el 02/12/2025.

Para obtener la hora dividimos lo que sobra de la resta anterior:

\begin{Shaded}
\begin{Highlighting}[]
\NormalTok{resto =}\StringTok{ }\NormalTok{d }\OperatorTok{-}\StringTok{ }\DecValTok{28}
\NormalTok{hora =}\StringTok{ }\NormalTok{resto }\OperatorTok{*}\StringTok{ }\DecValTok{24}
\NormalTok{minuto =}\StringTok{ }\NormalTok{(hora}\OperatorTok\KeywordTok{floor}\NormalTok{(hora)) }\OperatorTok{*}\StringTok{ }\DecValTok{60}
\NormalTok{segundo =}\StringTok{ }\NormalTok{(minuto}\OperatorTok\KeywordTok{floor}\NormalTok{(minuto)) }\OperatorTok{*}\StringTok{ }\DecValTok{60}
\NormalTok{hora =}\StringTok{ }\KeywordTok{floor}\NormalTok{(hora)}
\NormalTok{minuto =}\StringTok{ }\KeywordTok{floor}\NormalTok{(minuto)}
\NormalTok{segundo =}\StringTok{ }\KeywordTok{floor}\NormalTok{(minuto)}
\KeywordTok{print}\NormalTok{(}\KeywordTok{paste}\NormalTok{(}\StringTok{"La hora es"}\NormalTok{, hora, }\StringTok{":"}\NormalTok{, minuto, }\StringTok{":"}\NormalTok{ , segundo))}
\end{Highlighting}
\end{Shaded}

\begin{verbatim}
  [1] "La hora es 11 : 33 : 33"
\end{verbatim}

\hypertarget{pregunta-2}{%
\subsection{PREGUNTA 2:}\label{pregunta-2}}

Cread una función que os resuelva una ecuación de primer grado (de la
forma Ax+B=0). Es decir, vosotros tendréis que introducir como
parámetros los coeficientes (en orden) y la función os tiene que
devolver la solución. Por ejemplo, si la ecuación es 2x+4=0, vuestra
función os tendría que devolver -2.

\begin{Shaded}
\begin{Highlighting}[]
\NormalTok{res_ec =}\StringTok{ }\ControlFlowTok{function}\NormalTok{ (A, B, C)\{}
\NormalTok{  resultado =}\StringTok{ }\NormalTok{C}\OperatorTok{-}\NormalTok{B}\OperatorTok{/}\NormalTok{A}
  \KeywordTok{print}\NormalTok{(}\KeywordTok{paste}\NormalTok{(}\StringTok{"La función a resolver es:"}\NormalTok{, A,}\StringTok{"* x +"}\NormalTok{, B, }\StringTok{"="}\NormalTok{, C))}
  \KeywordTok{print}\NormalTok{(}\KeywordTok{paste}\NormalTok{(}\StringTok{"El resultado es: x ="}\NormalTok{, resultado))}
\NormalTok{\}}
\end{Highlighting}
\end{Shaded}

Probamos la función:

\begin{Shaded}
\begin{Highlighting}[]
\KeywordTok{res_ec}\NormalTok{(}\DecValTok{5}\NormalTok{, }\DecValTok{3}\NormalTok{, }\DecValTok{0}\NormalTok{)}
\end{Highlighting}
\end{Shaded}

\begin{verbatim}
  [1] "La función a resolver es: 5 * x + 3 = 0"
  [1] "El resultado es: x = -0.6"
\end{verbatim}

\begin{Shaded}
\begin{Highlighting}[]
\KeywordTok{res_ec}\NormalTok{(}\DecValTok{7}\NormalTok{, }\DecValTok{4}\NormalTok{, }\DecValTok{18}\NormalTok{)}
\end{Highlighting}
\end{Shaded}

\begin{verbatim}
  [1] "La función a resolver es: 7 * x + 4 = 18"
  [1] "El resultado es: x = 17.4285714285714"
\end{verbatim}

\begin{Shaded}
\begin{Highlighting}[]
\KeywordTok{res_ec}\NormalTok{(}\DecValTok{1}\NormalTok{, }\DecValTok{1}\NormalTok{, }\DecValTok{1}\NormalTok{)}
\end{Highlighting}
\end{Shaded}

\begin{verbatim}
  [1] "La función a resolver es: 1 * x + 1 = 1"
  [1] "El resultado es: x = 0"
\end{verbatim}

\hypertarget{pregunta-3}{%
\subsection{PREGUNTA 3}\label{pregunta-3}}

Dad una expresión para calcular 3e-π y a continuación, dad el resultado
que habéis obtenido con R redondeado a 3 cifras decimales.

\begin{Shaded}
\begin{Highlighting}[]
\NormalTok{exp1 =}\StringTok{ }\DecValTok{3}\OperatorTok{*}\KeywordTok{exp}\NormalTok{(}\OperatorTok{-}\NormalTok{pi)}
\NormalTok{exp1 =}\StringTok{ }\KeywordTok{round}\NormalTok{(exp1, }\DataTypeTok{digits=}\DecValTok{3}\NormalTok{)}
\NormalTok{exp1}
\end{Highlighting}
\end{Shaded}

\begin{verbatim}
  [1] 0.13
\end{verbatim}

Dad el módulo del número complejo (2+3i)\^{}2/(5+8i) redondeado a 3
cifras decimales.

\begin{Shaded}
\begin{Highlighting}[]
\NormalTok{complejo =}\StringTok{ }\NormalTok{(}\DecValTok{2}\OperatorTok{+}\NormalTok{3i)}\OperatorTok{^}\DecValTok{2}\OperatorTok{/}\NormalTok{(}\DecValTok{5}\OperatorTok{+}\NormalTok{8i)}
\KeywordTok{round}\NormalTok{(complejo, }\DataTypeTok{digits =} \DecValTok{3}\NormalTok{)}
\end{Highlighting}
\end{Shaded}

\begin{verbatim}
## [1] 0.798+1.124i
\end{verbatim}


\end{document}
